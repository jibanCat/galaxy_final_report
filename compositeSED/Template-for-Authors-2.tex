% template.tex, dated April 5 2013
% This is a template file for Annual Reviews 1 column Journals
%
% Compilation using ar-1col.cls' - version 1.0, Aptara Inc.
% (c) 2013 AR
%
% Steps to compile: latex latex latex
%
% For tracking purposes => this is v1.0 - Apr. 2013
\documentclass{ar-1col}
\usepackage{natbib}

\setcounter{secnumdepth}{4}
\usepackage{url}

% preferred fontsize
\newcommand{\mysize}{\fontsize{12}{14}\selectfont}

% Metadata Information
\jname{Xxxx. Xxx. Xxx. Xxx.}
\jvol{AA}
\jyear{YYYY}
\doi{10.1146/((please add article doi))}


% Document starts
\begin{document}

% Page header
\markboth{Ho}{Composite SED Clustering}

% Title
\title{Composite Spectral Energy Distribution and Clustering Methods}

%Authors, affiliations address.
\author{Ming-Feng Ho$^1$
\affil{$^1$Department of Physics and Astronomy, University of California, Riverside; email: mho026@ucr.edu}}

%Abstract
\begin{abstract}
Composite SEDs and Bayesian non-parametric clustering.

Photometry and medium bands: surveys
Spectral Energy Distributions: fitting template, FAST, EAZY
Composite SEDs: evolution from grouping methods
Bayesian non-parametric on functional data:
1. Dirichlet Processes for clustering
2. Gaussian Processes on Spectral data
3. Clustering on functional data 

\end{abstract}

%Keywords, etc.
\begin{keywords}
galaxies: evolution, methods: data analysis, methods: Bayesian non-parametric
\end{keywords}
\maketitle

%Table of Contents
\tableofcontents


% Heading 1
\section{INTRODUCTION}
improvement of Photometry data on redshift rage 3-4.


% Heading 2
\section{GALAXY EVOLUTION IN TERMS OF COMPOSITE SPECTRAL ENERGY DISTRIBUTIONS (SEDs)}

% Heading 2.1
\subsection{Medium-Band Photometry}
This is dummy text. 

% Heading 2.2
\subsection{Fitting Template of Spectral Energy Distributions (SEDs)}
This is dummy text. This is dummy text. This is dummy text. This is dummy text.

% Heading 2.3
\subsection{Composite Spectral Energy Distributions (SEDs)}

% Heading 3
\section{BAYESIAN MACHINE LEARNING FOR MODELING SPECTRAL DATA}

Bayesian machine learning is a branch of machine learning which aims to solve machine learning problems in a Bayesian perspective. Instead of optimizing the parameters of interest from data using an empirical loss function (e.g., a least-squared function), Bayesian methods build generative models to randomly sample data from parameters and try to maximize the likelihood between observed data and hidden parameters \citep{Barber2012}.

The difference between Bayesian statistics and Bayesian ``machine learning'' is that Bayesian ``machine learning'' is trying to approximate {\it non-linear} functions \citep{Bishop2003}. After the publish of \citet{Rasmussen2005}, learning unknown complicated functions from observed data using {\it Gaussian processes} (\textsc{gp}) became popular. 

A {\it Gaussian process} is a bunch of random variables, and any finite subset of these random variables is a joint Gaussian distribution \citep{Rasmussen2005}. \textsc{gp} could be a powerful tool to model any kind of functional data (continuous data) in a non-parametric way. By non-parametric, it actually means we use infinite many parameters to describe our function \citep{Gelman04}. \textsc{gp} could be treated as a random function (or a stochastic process) which draws samples from the n-dimensional distribution, 

\begin{equation}
    \mu(x_1), ..., \mu(x_n) \sim Normal((m(x_1), ..., m(x_n)), K(x_1, ..., x_n))
    \label{eq:GP}
\end{equation}



\citep{Garnett17}

\subsection{Gaussian Processes for Modeling Spectra}

\subsection{Bayesian Model Selection with Different Types of Galaxies}

\subsection{Dirichlet Processes for Clustering}

\subsection{Possibility: Dirichlet Processes combined with Gaussian Process for Modeling Composite SEDs}


% Summary Points
\begin{summary}[SUMMARY POINTS]
\begin{enumerate}
\item Summary point 1. These should be full sentences.
\end{enumerate}
\end{summary}

% Future Issues
\begin{issues}[FUTURE ISSUES]
\begin{enumerate}
\item Future issue 1. These should be full sentences.
\end{enumerate}
\end{issues}

%Disclosure
% \section*{DISCLOSURE STATEMENT}
% If the authors have noting to disclose, the following statement will be used: The authors are not aware of any affiliations, memberships, funding, or financial holdings that
% might be perceived as affecting the objectivity of this review. 

% Acknowledgements
\section*{ACKNOWLEDGMENTS}
Acknowledgements, general annotations, funding.

% References
\bibliographystyle{ar-style2}
\bibliography{sample}
% \begin{thebibliography}{00}

% \bibitem[Acevedo \& Fitzjarrald(2001)]{Acevedo:01}
% Acevedo O, Fitzjarrald D. 2001.
% \textit{J. Atmos. Sci.} 58:2650--67


% \end{thebibliography}


\end{document}
